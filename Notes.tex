\documentclass[11pt]{article}
\usepackage[utf8]{inputenc}

\begin{document}

\section{ECC Summer School 2015}
\subsection{Intro - Software and Hardware implementation of Elliptic Curve Cryptography}
Presentation by Jérémie Detrey.
Reminders on elliptic curves over finite fields, associated operations, ECDH and ElGamal. \newline
Jérémie then states the design goals of ECC implementations: efficient and secure. 

Efficiency can have multiple meanings. Does it refer to:
\begin{itemize}
	\item fast? $\to$ low latency or high throughput?
	\item small? $\to$ low memory / code / silicon usage?
	\item low power? ... or low energy?
\end{itemize}
Constraints are going to depend to the application and target platform.

The other design goal was security, but secure against which attack:
\begin{itemize}
	\item protocol attacks? (FREAK, LogJam, etc) [More in N. Heninger's talk]
	\item curve attacks? (weak curves, twist security, etc.)
	\item timing attacks? [More in P. Schwabe's talk]
	\item fault attacks? [More in J. Krämer's talk]
	\item cache attacks?
	\item branch-prediction attacks?
	\item power or electromagnetic analysis?
\end{itemize}

A list of possible target platforms is enumerated to highlight the discrepancy in resources between them.

The layer of ECC implementations are usually: 
\begin{itemize}
	\item protocol (OpenPGP, TLS, SSH, etc)
	\item cryptographic primitives (ECDH, ECDSA, etc)
	\item scalar multiplication
	\item elliptic curve arithmetic (point addition, point doubling, etc.)
	\item finite field arithmetic (addition, multiplication, inversion ...)
	\item native integer arithmetic (CPU instruction set)
\end{itemize}

For hardware implementations, there are additional layers:
\begin{itemize}
	\item logic circuits (registers, multiplexers, adders, etc.)
	\item logic gates (NOT, NAND, etc.) and wires
	\item transistors
\end{itemize}

All (top) layers might lead to critical vulnerabilities if poorly implemented! ECC is no more secure than its weakest link.

Jérémie concludes with some references. Most books are already quite old. He suggests to read proceedings of recent crypto conferences such as CHES.
\end{document}
